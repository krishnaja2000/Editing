\renewcommand{\theequation}{\theenumi}
\begin{enumerate}[label=\thesection.\arabic*.,ref=\thesection.\theenumi]
\numberwithin{equation}{enumi}

\item Position vectors of the quadrilateral have benn given as
Quadrilateral ABCD is formed by 
\begin{figure}[!ht]
\centering
\resizebox{\columnwidth}{!}{\input{./figs/ques.tex}}
\caption{Chords of the circle intersecting at point B by Latex-Tikz}
\label{fig:intersecting_chords}	
\end{figure}

\item Show that angle subtended by a chord at the centre is twice the angle subtended by it at any point on the circumference
%figure for proof
\begin{figure}[!ht]
\centering
\resizebox{\columnwidth}{!}{\input{./figs/proof.tex}}
\caption{Angles subtended by chord ED at centre O and point C by Latex-Tikz}
\label{fig:chord_proof}	
\end{figure}
Solution:
\newline
In the figure \ref{fig:chord_proof}	
\begin{align}
OC = OD = OE = r
\end{align}
\implies $\triangle DOE$, $\triangle DOC$, $\triangle COE$ are isosceles triangle.
\newline
Hence their base angles are equal as indicated in figure \ref{fig:chord_proof}

\item In $\triangle EDC$ in figure \ref{fig:chord_proof} , We know sum of the angles in a triangle is $180^{\circ}$
\begin{align}
2(x+y+z) = 180^{\circ}
\\
180^{\circ} - 2y = 2(x+z)
\label{eq:val2y}
\end{align}

\item In $\triangle EOD$ in figure \ref{fig:chord_proof} as sum of the angles in a triangle is $180^{\circ}$,
\begin{align}
\angle EOD = 180^{\circ} - 2y
\label{eq:angle1proof}
\end{align}
Substituting equation \ref{eq:val2y} in equation \ref{eq:angle1proof} we get 
\begin{align}
\angle EOD = 2(x+z)
\\
\angle EOD = 2 \angle ECD
\\
\angle ECD = frac{\angle EOD}{2} = frac{\alpha}{2}
\label{eq:angle1}
\end{align}
Hence angle subtended by the chord at the centre is twice the angle subtended by it at a point on the circumference.

\item From \ref{fig:intersecting_chords} AC subtends $\angle AOC = \beta$ at the centre and $\angle ADC$ at the circumference.
\begin{align}
\implies \angle ADC = \frac{\angle AOB}{2}
\\
\implies \angle ADC = \frac{\beta}{2}
\label{eq:angle2}
\end{align}


\item In $\triangle DBC$ from \ref{eq:angle1} and \ref{eq:angle2}
\begin{align}
\angle BCD = 180^{\circ} - \angle ECD
\\
\angle BCD = 180^{\circ} - frac{\alpha}{2}
\label{eq:tri1}
\\
\angle BDC = \angle ADC = \frac{\beta}{2}
\label{eq:tri2}
\\
\angle CBD = \angle CBA = \theta
\label{eq:tri3}
\end{align}

Sum of the angles in a triangle is 180^{\circ}. So In $\triangle DBC$
\begin{align}
\angle CBD + \angle BDC + \angle BCD = 180^{\circ}
\label{eq:trisum}
\end{align}
Substituting equations \ref{eq:tri3},\ref{eq:tri3},\ref{eq:tri3} in \ref{eq:trisum}
\begin{align}
\theta + \frac{\beta}{2} + 180^{\circ} - frac{\alpha}{2} = 180^{\circ}
\\
\theta = frac{\alpha}{2} - \frac{\beta}{2}
\\
\theta = frac{\alpha - \beta}{2} 
\end{align}
Hence proved.
\end{enumerate}
