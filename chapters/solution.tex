\renewcommand{\theequation}{\theenumi}
\begin{enumerate}[label=\thesection.\arabic*.,ref=\thesection.\theenumi]
\numberwithin{equation}{enumi}

\item A conic section has the following equation\ref{eq:alg_conic}
\begin{align}
Ax^2+Bxy+Cy^2+Dx+Ey+F=0
\label{eq:alg_conic}
\end{align}
The equation is expressed in vector form is as follows
The vector form is 
\begin{align}
\label{eq:vec_conic}
\vec{x}^T \myvec{A & B/2\\B/2 & C}\vec{x}+ \myvec{D & E}\vec{x}+F=0
\end{align}

\item 
\begin{align}
\label{eq:alg_conic_a}
\brak{a} 12x^{2} - 7x + 1
\end{align}
can be expressed according to \ref{eq:vec_conic} as 
\begin{align}
\label{eq:vec_conic_a}
\vec{x}^T \myvec{12 & 0\\0 & 0}\vec{x}+ \myvec{-7 & 0}\vec{x}+1=0
\end{align}
If $\myvec{k//0}$ satisfies \ref{eq:vec_conic_a} then \brak{x-k} is a factor of \ref{eq:alg_conic_a}
\begin{align}
\implies 
12k^{2}-7k+1=0
\\
\brak{3k-1}\brak{4k-1}=0
\\
k=\frac{1}{3}, \frac{1}{4}
\\
\brak{x-\frac{1}{3}}\brak{x-\frac{1}{4}}=12x^{2} - 7x + 1
\end{align}

The following code sketches the graph of \ref{eq:alg_conic_a} in figure \ref{fig:conic2a}
\begin{lstlisting}
codes/conic2/conic2a.py
\end{lstlisting}
\begin{figure}[!ht]
\centering
\includegraphics[width=\columnwidth]{./codes/conic2/pyfigs/conic2a.eps}
\caption{Graph of \ref{eq:alg_conic_a}}
\label{fig:conic2a}
\end{figure}


\item 
\begin{align}
\label{eq:alg_conic_b}
\brak{b} 6x^{2} + 5x -6
\end{align}
can be expressed according to \ref{eq:vec_conic} as 
\begin{align}
\label{eq:vec_conic_b}
\vec{x}^T \myvec{6 & 0\\0 & 0}\vec{x}+ \myvec{5 & 0}\vec{x} -6 =0
\end{align}
If $\vec{x}=\myvec{k//0}$ satisfies \ref{eq:vec_conic_b} then \brak{x-k} is a factor of \ref{eq:alg_conic_b}
\begin{align}
\implies 
6k^{2}+5k-6=0
\\
\brak{2k+3}\brak{3k-2}=0
\\
k=\frac{-3}{2}, \frac{2}{3}
\\
c\brak{x-\frac{-3}{2}}\brak{x-\frac{2}{3}}= 6x^{2} + 5x -6
\\
\brak{2x+3}\brak{3x-2}= 6x^{2} + 5x -6
\end{align}
The following code sketches the graph of \ref{eq:alg_conic_b} in figure \ref{fig:conic2b}
\begin{lstlisting}
codes/conic2/conic2b.py
\end{lstlisting}
\begin{figure}[!ht]
\centering
\includegraphics[width=\columnwidth]{./codes/conic2/pyfigs/conic2b.eps}
\caption{Graph of \ref{eq:alg_conic_b}}
\label{fig:conic2b}
\end{figure}

\item 
\begin{align}
\label{eq:alg_conic_c}
\brak{c} 2x^{2} + 7x +3
\end{align}
can be expressed according to \ref{eq:vec_conic} as 
\begin{align}
\label{eq:vec_conic_c}
\vec{x}^T \myvec{2 & 0\\0 & 0}\vec{x}+ \myvec{7 & 0}\vec{x} +3 =0
\end{align}
If $\vec{x}=\myvec{k//0}$ satisfies \ref{eq:vec_conic_c} then \brak{x-k} is a factor of \ref{eq:alg_conic_c}
\begin{align}
\implies 
2k^{2}+7k+3=0
\\
\brak{2k+1}\brak{k+3}=0
\\
k=\frac{-1}{2},-3
\\
\brak{2x+1}\brak{x+3}= 2x^{2} + 7x +3
\end{align}

The following code sketches the graph of \ref{eq:alg_conic_c} in figure \ref{fig:conic2c}
\begin{lstlisting}
codes/conic2/conic2c.py
\end{lstlisting}
\begin{figure}[!ht]
\centering
\includegraphics[width=\columnwidth]{./codes/conic2/pyfigs/conic2c.eps}
\caption{Graph of \ref{eq:alg_conic_c}}
\label{fig:conic2c}
\end{figure}


\item 
\begin{align}
\label{eq:alg_conic_d}
\brak{d} 3x^{2} -x -4
\end{align}
can be expressed according to \ref{eq:vec_conic} as 
\begin{align}
\label{eq:vec_conic_d}
\vec{x}^T \myvec{3 & 0\\0 & 0}\vec{x}+ \myvec{-1 & 0}\vec{x} -4 =0
\end{align}
If $\vec{x}=\myvec{k//0}$ satisfies \ref{eq:vec_conic_d} then \brak{x-k} is a factor of \ref{eq:alg_conic_d}
\begin{align}
\implies 
3k^{2}-k-4=0
\\
\brak{3k-4}\brak{k+1}=0
\\
k=\frac{4}{3},-1
\\
\brak{3x-4}\brak{x+1}= 3x^{2} -x -4
\end{align}

The following code sketches the graph of \ref{eq:alg_conic_d} in figure \ref{fig:conic2d}
\begin{lstlisting}
codes/conic2/conic2d.py
\end{lstlisting}
\begin{figure}[!ht]
\centering
\includegraphics[width=\columnwidth]{./codes/conic2/pyfigs/conic2d.eps}
\caption{Graph of \ref{eq:alg_conic_d}}
\label{fig:conic2d}
\end{figure}


\end{enumerate}
