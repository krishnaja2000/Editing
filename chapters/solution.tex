\renewcommand{\theequation}{\theenumi}
\begin{enumerate}[label=\thesection.\arabic*.,ref=\thesection.\theenumi]
\numberwithin{equation}{enumi}

\item $\vec{x}=\myvec{x\\y}$ id the solution, Then for every value of x , there is corresponding value of y. 
Let $\myvec{0\\a}$ and $\myvec{1\\b}$ be the two solutions of equation \ref{eq:pointonline1}.Then
\begin{align}
 \myvec{4&3}\myvec{0\\a}=12
 \implies a=4
 \\
 \myvec{4&3}\myvec{1\\b}=12
 \implies y=\frac{8}{3}
\end{align}
So $\myvec{0\\4}$ and $\myvec{1\\\frac{8}{3}}$ are the two solutions of \ref{eq:pointonline1}

\item 
Let $\myvec{0\\a}$ and $\myvec{1\\b}$ be the two solutions of equation \ref{eq:pointonline2}.Then
\begin{align}
 \myvec{2&5}\myvec{0\\a}=0
 \implies a=0
 \\
 \myvec{2&5}\myvec{1\\b}=0
 \implies b=\frac{-2}{5}
\end{align}

So $\myvec{0\\0}$ and $\myvec{1\\\frac{-2}{5}}$ are the two solutions of \ref{eq:pointonline2}
\item 
Let $\myvec{x\\0}$ and $\myvec{0\\y}$ be the two solutions of equation \ref{eq:pointonline1}.Then
\begin{align}
 \myvec{4&3}\myvec{x\\0}=12
 \implies x=3
 \\
 \myvec{4&3}\myvec{0\\y}=12
 \implies y=4
\end{align}
So $\myvec{3\\0}$ and $\myvec{0\\4}$ are the two solutions of \ref{eq:pointonline1}
Let $\vec{C}$ divide AB in ratio k:1.Then by section formulae,
\begin{align}
\vec{C}=\frac{k\vec{B}+\vec{A}}{k+1}
\\
\myvec{-1\\6}=\frac{1}{k+1}\myvec{6k-3\\-8k+10}
\\
k=\frac{2}{7}
\end{align}
So $\vec{C}$ divides AB in ratio 2:7
\newline
The following Python code generates Fig. \ref{fig:section}
%
\begin{lstlisting}
codes/section/section.py
\end{lstlisting}
\begin{figure}[!ht]
\centering
\includegraphics[width=\columnwidth]{./codes/section/pyfigs/section.eps}
\caption{C divides AB in ratio k:1}
\label{fig:section}
\end{figure}
\end{enumerate}
