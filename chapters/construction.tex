%
\renewcommand{\theequation}{\theenumi}
\begin{enumerate}[label=\thesection.\arabic*.,ref=\thesection.\theenumi]
\numberwithin{equation}{enumi}

\item The two chords AC and DA of circle with centre $\vec{O}$ and radiusr intersect at \vec{B} as shown in the figure \ref{fig:circle.tex}

%\renewcommand{\thefigure}{\theenumi.\arabic{figure}}
\begin{figure}[!ht]
\centering
\resizebox{\columnwidth}{!}{\input{./figs/circle.tex}}
\caption{Chords of equal length intersecting at B by Latex-Tikz}
\label{fig:circle}	
\end{figure}
%


%
%\renewcommand{\thefigure}{\theenumi}
%
\item List the design parameters for construction
\label{const:table1}
\\
\solution See Table. \ref{table:table1} 
%
\begin{table}[ht!]
\centering
\input{./tables/inp.tex}
\caption{To construct circle with chords intersecting at an external point}
\label{table:table1}	
\end{table}

%\item

\item Find the coordinates of the various points in Fig. \ref{fig:circle}
\label{const:circle_r}
\\
%
\solution $\vec{O}$ is taken to be the origin and OC is taken to be along the x-axis.
\begin{align}
\label{eq:constr_o}
\vec{O} &= \myvec{0\\0} 
\\
\vec{C} &= \myvec{r\\0} 
\label{eq:constr_c}
\end{align}

\item Any point X lying on the circle satisfies the following equation
 \begin{align}
r &=\norm{\vec{X}-\vec{O}}
\\
r &=\norm{\vec{X}}
\\
\vec{X} &= \myvec{\cos \theta \\ \sin \theta}
\label{eq:point_on_circle}
\end{align}
Where $\theta$ is taken to be the angle measured in anti-clockwise direction from x-axis.

\item Using \ref{eq:point_on_circle}
\begin{align}
\implies \vec{A} &= \myvec{\cos \beta \\ -\sin \beta}
\label{eq:constr_a}
\end{align}
%

\item In $\triangle AOD$ and $\triangle COE$
\begin{align}
OD = OE = r
OC = OA = r
AD = EC 
\end{align}
So by SSS criteria $\triangle AOD \cong \triangle COE$
\begin{align}
\angle EOC = \angle DOA
\\
\angle EOC = \angle DOA = \frac{360^{\circ}-\brak{\alpha+\beta}}{2}
\\
\angle EOC = \angle DOA = 180^{\circ}-\frac{\alpha+\beta}{2}
\label{eq:equalangles}
\end{align}

\item Using \ref{eq:point_on_circle} and \ref{eq:equalangles}
\begin{align}
\vec{D} &= \myvec{\cos \brak{180^{\circ}-\frac{\alpha+\beta}{2}+\alpha} \\ \sin \brak{180^{\circ}-\frac{\alpha+\beta}{2}+\alpha}} 
\\
\vec{D} &= \myvec{\cos \brak{180^{\circ}+\frac{\alpha-\beta}{2}} \\ \sin \brak{180^{\circ}+\frac{\alpha-\beta}{2}}}
\\
\vec{D} &= \myvec{-\cos \frac{\alpha-\beta}{2} \\ -\sin \frac{\alpha-\beta}{2}}
\label{eq:constr_d}
\end{align}
\begin{align}
\vec{E} &= \myvec{\cos \brac{180^{\circ}-\frac{\alpha+\beta}{2}} \\ \sin \brac{180^{\circ}-\frac{\alpha+\beta}{2}}}
\\
\vec{E} &= \myvec{-\cos \frac{\alpha+\beta}{2} \\ \sin \frac{\alpha+\beta}{2}}

\end{align}


The derived coordinates are listed in \ref{table:table2}. 
%
\begin{table}[ht!]
\centering
\input{./tables/out.tex}
\caption{Derived coordinates}
\label{table:table2}	
\end{table}
\item Draw Fig. \ref{fig:circle} using python	
\\
\solution The  following Python code generates Fig. \ref{fig:circle}
%
\begin{lstlisting}
codes/circle.py
\end{lstlisting}
\begin{figure}[!ht]
\centering
\includegraphics[width=\columnwidth]{./codes/pyfigs/circle.eps}
\caption{Circle with two chords intersecting at external point B}
\label{fig:circle_py}
\end{figure}

%
and the equivalent latex-tikz code generating Fig. \ref{fig:trapezium_ABCD} is 
\begin{lstlisting}
figs/circle.tex
\end{lstlisting}
%
The above latex code can be compiled as a standalone document as
\begin{lstlisting}
figs/circle_alone.tex
\end{lstlisting}

%

\end{enumerate}
