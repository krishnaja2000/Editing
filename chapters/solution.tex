\renewcommand{\theequation}{\theenumi}
\begin{enumerate}[label=\thesection.\arabic*.,ref=\thesection.\theenumi]
\numberwithin{equation}{enumi}

\item Let $\vec{O}$ be the centre , r be the radius of the circle.Any point $\vec{X}$ lying on the circle is at a distance r from $\vec{O}$.
\newline
Therefore the equation of the circle is 
\begin{align}
\norm{\vec{X}-\vec{O}} &=r
\label{eq:eqn_of_circle}
\end{align}


The following code sketches the circles in figure\ref{fig:circle2} using the equation \ref{eq:eqn_of_circle}
\begin{lstlisting}
codes/circle2/circle2.py.py
\end{lstlisting}
\begin{figure}[!ht]
\centering
\includegraphics[width=\columnwidth]{./codes/circle2/pyfigs/circle2.eps}
\caption{Circles with centre at $\vec{O_n}$ and radius $r_n$}
\label{fig:circle2}
\end{figure}
The parameters to construct a circle of centre $\myvec{-a\\-b}$, radius $\sqrt{a^2-b^2}$ are considered as $a=5$ and $b=4$.
\end{enumerate}
